%----------------------------------------------------------------------------------------
%	PACKAGES AND OTHER DOCUMENT CONFIGURATIONS
%----------------------------------------------------------------------------------------

\documentclass[11pt]{scrartcl} % Font size
\input{structure.tex} % Include the file specifying the document structure and custom commands

%----------------------------------------------------------------------------------------
%	TITLE SECTION
%----------------------------------------------------------------------------------------

\title{
	\normalfont\normalsize
	\textsc{Harvard Privacy Tools Project}\\ % Your university, school and/or department name(s)
	\vspace{25pt} % Whitespace
	\rule{\linewidth}{0.5pt}\\ % Thin top horizontal rule
	\vspace{20pt} % Whitespace
	{\huge Reasoning About Utility of the Snapping Mechanism}\\ % The assignment title
	\vspace{12pt} % Whitespace
	\rule{\linewidth}{2pt}\\ % Thick bottom horizontal rule
	\vspace{12pt} % Whitespace
}

\author{\LARGE Christian Covington} % Your name

\date{\normalsize\today} % Today's date (\today) or a custom date

\begin{document}
\maketitle % Print the title

\section{Snapping Mechanism Introduction}
The snapping mechanism is a differentially-private mechanism introduced in \href{http://citeseerx.ist.psu.edu/viewdoc/download?doi=10.1.1.366.5957&rep=rep1&type=pdf}{Mironov (2012)} that avoids the vulnerabilities\footnote{due to sampling from floating-point numbers rather than the reals} of the Laplace mechanism that were first demonstrated in the same paper. Our goal is to assess (both empirically and theoretically) the utility loss of using the snapping mechanism as opposed to the Laplace mechanism. \newline

\section{Definitions}
\subsection{Laplace Mechanism}
Let $f$ be a function computed on a dataset $D$ with sensitivity $\delta f$ and $\epsilon$ be the desired privacy parameter. Further, let $\lambda = \frac{\delta f}{\epsilon}$ the Laplace Mechanism is defined as:
\[ M_{L}(D, f(\cdot), \lambda) = f(D) + Y \]
where $Y \sim Laplace(\lambda)$.

\subsection{Snapping Mechanism}
Let $B$ be a user-chosen quantity that reflects beliefs about reasonable bounds on $f(D)$ and $\Lambda$ be the smallest power of two at least as large as $\lambda$. Using the same notation as above, the snapping mechanism is defined as:
\[ M_{S}(D, f(\cdot), \lambda, B) = clamp_{B}\left( \lfloor clamp_{B}\left( f(D) \right) + Y \rceil_{\Lambda} \right). \]
where $clamp_{B}(\cdot)$ restricts output to the interval $[-B, B]$ and $\lfloor \cdot \rceil_{\Lambda}$ rounds to the nearest multiple of $\Lambda$, with ties resolved toward $+ \infty$. We will also make use of $\lfloor \cdot \rceil^{*}_{\Lambda}$, which rounds to the nearest multiple of $\Lambda$, with ties resolved toward $- \infty$. \newline

\subsection{Utility}
This remains undefined for now.

\section{Mechanism Comparison}
\subsection{Distance}
We first examine the distance between output from the Laplace and Snapping Mechanisms. We define the distance as follows:
\[ d_{M_L, M_S}(D, f(\cdot), \lambda, B) = \big\vert M_L(\cdot) - M_S(\cdot) \big\vert \]
which we will refer to as $dist$ for ease of notation. \newline

Let $D, f, \lambda$ be fixed. We will assume that $f(D) > 0$,
but letting $f(D) \leq 0$ changes only the strictness of equality at certain points.\footnote{I think $\hdots$} We consider different circumstances with respect to $B$:

\subsubsection{Case 1: $\boldsymbol{clamp_B}$ never binding}
Consider the case in which both $f(D)$ and $\lfloor f(D) + Y \rceil_{\Lambda}$ lie in the interval $[-B, B]$.
% Roughly, this will hold when the user chooses a relatively conservative bound, $B$, and $\lambda$ is small relative to $\vert f(D) - B \vert$.
Then $clamp_B$ is never binding and we can ignore it. Then we have:
\[ dist_1 = \big\vert f(D) + Y - \lfloor f(D) + Y \rceil_{\Lambda} \big\vert \]
We know that $f(D) + Y$ is, at most, $\frac{\Lambda}{2}$ away from the nearest multiple of $\Lambda$. Furthermore, we have $\Lambda = 2^m$ for some $m \in \mathbb{Z}$ so that $2^{m-1} < \lambda \leq 2^{m}$. Thus, we know:
\begin{align}
	dist_1 &= \big\vert f(D) + Y - \lfloor f(D) + Y \rceil_{\Lambda} \big\vert \nonumber \\
		 &\leq \frac{\Lambda}{2} \nonumber \\
		 &= \frac{2^m}{2} \nonumber \\
		 &= 2^{m-1} \nonumber \\
		 &< \lambda. \nonumber
\end{align}

We can now consider the expectation.

\subsubsection{Case 2: inner $\boldsymbol{clamp_B}$ binding}
Consider the case in which $f(D) \not\in [-B, B]$ but $\lfloor clamp_{B}\left( f(D) \right) + Y \rceil_{\Lambda} \in [-B, B]$.
% This corresponds to the scenario in which the observed statistic is greater than the user's proposed bound and $B,Y$ are such that $\lfloor B + Y \rceil_{\Lambda}$ rounds toward 0.
Because we assume $f(D) > 0$, we know that $f(D) > B$. We keep the inner clamp and ignore the outer and get:
\begin{align}
	dist_2 &= \big\vert f(D) + Y - \lfloor clamp_{B}\left(f(D)\right) + Y \rceil_{\Lambda} \big\vert \nonumber \\
		   &= \big\vert f(D) + Y - \lfloor B + Y \rceil_{\Lambda} \big\vert \nonumber
\end{align}
Similar to Case 1, we know that $\lfloor B + Y \rceil_{\Lambda}$ is at most $\frac{\Lambda}{2}$ away from $B + Y$. More precisely, we know
\[ B + Y - \frac{\Lambda}{2} < \lfloor B + Y \rceil_{\Lambda} \leq B + Y + \frac{\Lambda}{2} \]
so we can use this to bound $dist_2$:
\begin{align}
	dist_2 &= \big\vert f(D) + Y - \lfloor B + Y \rceil_{\Lambda} \big\vert \nonumber \\
		   &< \bigg\vert f(D) + Y - \left( B + Y - \frac{\Lambda}{2} \right) \bigg\vert \nonumber \\
		   &= \bigg\vert f(D) - B + \frac{\Lambda}{2} \bigg\vert \nonumber \\
		   &= f(D) - B + \frac{\Lambda}{2} \nonumber \\
		   &< f(D) - B + \lambda \nonumber
\end{align}

\subsubsection{Case 3: outer $\boldsymbol{clamp_B}$ binding}
Now we consider the case in which $f(D) \in [-B, B]$ but $\lfloor f(D) + Y \rceil_{\Lambda} \not\in [-B, B]$. We know then that $clamp_{B} \left( \lfloor f(D) + Y \rceil_{\Lambda} \right) \in \{-B, B\}$. So we have:
\begin{align}
	dist_3 &= \bigg\vert f(D) + Y - clamp_{B}\left( \lfloor f(D) + Y \rceil_{\Lambda} \right) \bigg\vert \nonumber \\
		   &= \bigg\vert f(D) + Y \pm B \bigg \vert \nonumber
\end{align}
This is technically unbounded because $Y$ is unbounded, so we should probably return to this and try another approach (e.g. getting the expectation or bounding with high probability).

\subsubsection{Case 4: both $\boldsymbol{clamp_B}$ binding}
Finally, we consider the case in which $f(D) \not\in [-B,B]$ and $\lfloor clamp_{B}\left( f(D) \right) + Y \rceil_{\Lambda} \not\in [-B, B]$. Because $f(D) > 0$, we know that $clamp_{B} \left( f(D) \right) = B$. So,
\begin{align}
	dist_4 &= \big\vert f(D) + Y - clamp_{B} \left( \lfloor clamp_{B} \left( f(D) \right)  + Y \rceil_{\Lambda} \right) \big\vert \nonumber \\
		   &= \big\vert f(D) + Y - clamp_{B} \left( \lfloor B + Y \rceil_{\Lambda} \right) \big\vert \nonumber \\
		   &= \big\vert f(D) + Y \pm B \big\vert \nonumber
\end{align}
As for Case 3, this is unbounded.

% NOTE: Maybe useful to know for expectation --\vert laplace(\lambda) \vert ~ exponential(\lambda^{-1})
%       expectation of exponential(\lambda^{-1}) is \lambda^{-1}
%
% Could think of Y ~ laplace(\lambda) as being censored (call censored version Y') s.t. f(D) + Y' \in [-B,B], but this means that Y' is a function of f(D), but can still use linearity of expectation
%
% Need to figure out how to get expectation of censored (not truncated) random variable

\end{document}



















\begin{comment}
%
% ----------------------------------------------------------------------------------------
% 	FIGURE EXAMPLE
% ----------------------------------------------------------------------------------------
%
% \section{Image Interpretation}
%
% \begin{figure}[h] % [h] forces the figure to be output where it is defined in the code (it suppresses floating)
% 	\centering
% 	\includegraphics[width=0.5\columnwidth]{swallow.jpg} % Example image
% 	\caption{European swallow.}
% \end{figure}
%
% ------------------------------------------------
%
% \subsection{What is the airspeed velocity of an unladen swallow?}
%
% While this question leaves out the crucial element of the geographic origin of the swallow, according to Jonathan Corum, an unladen European swallow maintains a cruising airspeed velocity of \textbf{11 metres per second}, or \textbf{24 miles an hour}. The velocity of the corresponding African swallows requires further research as kinematic data is severely lacking for these species.
%
% %----------------------------------------------------------------------------------------
% %	TEXT EXAMPLE
% %----------------------------------------------------------------------------------------

\section{Understanding Text}

\subsection{How much wood would a woodchuck chuck if a woodchuck could chuck wood?}

%------------------------------------------------

\subsubsection{Suppose ``chuck" implies throwing.}

According to the Associated Press (1988), a New York Fish and Wildlife technician named Richard Thomas calculated the volume of dirt in a typical 25--30 foot (7.6--9.1 m) long woodchuck burrow and had determined that if the woodchuck had moved an equivalent volume of wood, it could move ``about \textbf{700 pounds (320 kg)} on a good day, with the wind at his back".

%------------------------------------------------

\subsubsection{Suppose ``chuck" implies vomiting.}

A woodchuck can ingest 361.92 cm\textsuperscript{3} (22.09 cu in) of wood per day. Assuming immediate expulsion on ingestion with a 5\% retainment rate, a woodchuck could chuck \textbf{343.82 cm\textsuperscript{3}} of wood per day.

%------------------------------------------------

\paragraph{Bonus: suppose there is no woodchuck.}

Fusce varius orci ac magna dapibus porttitor. In tempor leo a neque bibendum sollicitudin. Nulla pretium fermentum nisi, eget sodales magna facilisis eu. Praesent aliquet nulla ut bibendum lacinia. Donec vel mauris vulputate, commodo ligula ut, egestas orci. Suspendisse commodo odio sed hendrerit lobortis. Donec finibus eros erat, vel ornare enim mattis et.

%----------------------------------------------------------------------------------------
%	EQUATION EXAMPLES
%----------------------------------------------------------------------------------------

\section{Interpreting Equations}

\subsection{Identify the author of Equation \ref{eq:bayes} below and briefly describe it in English.}

\begin{align}
% 	\label{eq:bayes}
% 	\begin{split}
% 		P(A|B) = \frac{P(B|A)P(A)}{P(B)}
% 	\end{split}
% \end{align}
%
% Lorem ipsum dolor sit amet, consectetur adipiscing elit. Praesent porttitor arcu luctus, imperdiet urna iaculis, mattis eros. Pellentesque iaculis odio vel nisl ullamcorper, nec faucibus ipsum molestie. Sed dictum nisl non aliquet porttitor. Etiam vulputate arcu dignissim, finibus sem et, viverra nisl. Aenean luctus congue massa, ut laoreet metus ornare in. Nunc fermentum nisi imperdiet lectus tincidunt vestibulum at ac elit. Nulla mattis nisl eu malesuada suscipit.
%
% %------------------------------------------------
%
% \subsection{Try to make sense of some more equations.}
%
% \begin{align}
% 	\begin{split}
% 		(x+y)^3 &= (x+y)^2(x+y)\\
% 		&=(x^2+2xy+y^2)(x+y)\\
% 		&=(x^3+2x^2y+xy^2) + (x^2y+2xy^2+y^3)\\
% 		&=x^3+3x^2y+3xy^2+y^3
% 	\end{split}
% \end{align}
%
% Lorem ipsum dolor sit amet, consectetuer adipiscing elit.
% \begin{align}
% 	A =
% 	\begin{bmatrix}
% 		A_{11} & A_{21} \\
% 		A_{21} & A_{22}
% 	\end{bmatrix}
% \end{align}
% Aenean commodo ligula eget dolor. Aenean massa. Cum sociis natoque penatibus et magnis dis parturient montes, nascetur ridiculus mus. Donec quam felis, ultricies nec, pellentesque eu, pretium quis, sem.
%
% %----------------------------------------------------------------------------------------
% %	LIST EXAMPLES
% %----------------------------------------------------------------------------------------
%
% \section{Viewing Lists}
%
% \subsection{Bullet Point List}
%
% \begin{itemize}
% 	\item First item in a list
% 		\begin{itemize}
% 		\item First item in a list
% 			\begin{itemize}
% 			\item First item in a list
% 			\item Second item in a list
% 			\end{itemize}
% 		\item Second item in a list
% 		\end{itemize}
% 	\item Second item in a list
% \end{itemize}
%
% %------------------------------------------------
%
% \subsection{Numbered List}
%
% \begin{enumerate}
% 	\item First item in a list
% 	\item Second item in a list
% 	\item Third item in a list
% \end{enumerate}
%
% %----------------------------------------------------------------------------------------
% %	TABLE EXAMPLE
% %----------------------------------------------------------------------------------------
%
% \section{Interpreting a Table}
%
% \begin{table}[h] % [h] forces the table to be output where it is defined in the code (it suppresses floating)
% 	\centering % Centre the table
% 	\begin{tabular}{l l l}
% 		\toprule
% 		\textit{Per 50g} & \textbf{Pork} & \textbf{Soy} \\
% 		\midrule
% 		Energy & 760kJ & 538kJ\\
% 		Protein & 7.0g & 9.3g\\
% 		Carbohydrate & 0.0g & 4.9g\\
% 		Fat & 16.8g & 9.1g\\
% 		Sodium & 0.4g & 0.4g\\
% 		Fibre & 0.0g & 1.4g\\
% 		\bottomrule
% 	\end{tabular}
% 	\caption{Sausage nutrition.}
% \end{table}
%
% %------------------------------------------------
%
% \subsection{The table above shows the nutritional consistencies of two sausage types. Explain their relative differences given what you know about daily adult nutritional recommendations.}
%
% Lorem ipsum dolor sit amet, consectetur adipiscing elit. Praesent porttitor arcu luctus, imperdiet urna iaculis, mattis eros. Pellentesque iaculis odio vel nisl ullamcorper, nec faucibus ipsum molestie. Sed dictum nisl non aliquet porttitor. Etiam vulputate arcu dignissim, finibus sem et, viverra nisl. Aenean luctus congue massa, ut laoreet metus ornare in. Nunc fermentum nisi imperdiet lectus tincidunt vestibulum at ac elit. Nulla mattis nisl eu malesuada suscipit.
%
% %----------------------------------------------------------------------------------------
% %	CODE LISTING EXAMPLE
% %----------------------------------------------------------------------------------------
%
% % \section{Reading a Code Listing}
% %
% % \lstinputlisting[
% % 	caption=Luftballons Perl Script., % Caption above the listing
% % 	label=lst:luftballons, % Label for referencing this listing
% % 	language=Perl, % Use Perl functions/syntax highlighting
% % 	frame=single, % Frame around the code listing
% % 	showstringspaces=false, % Don't put marks in string spaces
% % 	numbers=left, % Line numbers on left
% % 	numberstyle=\tiny, % Line numbers styling
% % 	]{luftballons.pl}
%
% %------------------------------------------------
%
% \subsection{How many luftballons will be output by the Listing \ref{lst:luftballons} above?}
%
% Aliquam arcu turpis, ultrices sed luctus ac, vehicula id metus. Morbi eu feugiat velit, et tempus augue. Proin ac mattis tortor. Donec tincidunt, ante rhoncus luctus semper, arcu lorem lobortis justo, nec convallis ante quam quis lectus. Aenean tincidunt sodales massa, et hendrerit tellus mattis ac. Sed non pretium nibh. Donec cursus maximus luctus. Vivamus lobortis eros et massa porta porttitor.
%
% %------------------------------------------------
%
% \subsection{Identify the regular expression in Listing \ref{lst:luftballons} and explain how it relates to the anti-war sentiments found in the rest of the script.}
%
% Fusce varius orci ac magna dapibus porttitor. In tempor leo a neque bibendum sollicitudin. Nulla pretium fermentum nisi, eget sodales magna facilisis eu. Praesent aliquet nulla ut bibendum lacinia. Donec vel mauris vulputate, commodo ligula ut, egestas orci. Suspendisse commodo odio sed hendrerit lobortis. Donec finibus eros erat, vel ornare enim mattis et.
%
% %----------------------------------------------------------------------------------------
\end{comment}
